%% Technical Report for the work on the AI-DSL over 2024

\documentclass[]{report}
\usepackage{url}
\usepackage[utf8]{inputenc}
\usepackage{minted}
\usepackage[textsize=footnotesize]{todonotes}
\newcommand{\kabir}[2][]{\todo[color=yellow,author=kabir, #1]{#2}}
\newcommand{\nil}[2][]{\todo[color=purple,author=nil, #1]{#2}}
\usepackage[hyperindex,breaklinks]{hyperref}
\usepackage{breakurl}
\usepackage{listings}
\lstset{basicstyle=\ttfamily\footnotesize,breaklines=false,frame=single}
\usepackage{float}
\restylefloat{table}
\usepackage{longtable}
\usepackage{graphicx}
\usepackage[font=small,labelfont=bf]{caption}
\usepackage[skip=0pt]{subcaption}
\usepackage{circledsteps}

\begin{document}

\title{High Level Guide of AS-MOSES port to MeTTa}
\author{Nil Geisweiller}
\maketitle

\begin{abstract}
This document is a high level guide of the AS-MOSES port from OpenCog
Classic to MeTTa.  It also discusses the notion of Cognitive Synergy
in the context of MOSES as well as in the broader context.
\end{abstract}

\tableofcontents

\chapter{Introduction}

As the title suggests, this document is a high level guide of the
AS-MOSES port from OpenCog Classic to MeTTa.  It is primarily
addressed to the iCog team in charge of porting AS-MOSES to MeTTa.  In
fact this very document exists because Yeabsera Derese, currently the
lead of the AS-MOSES port project, asked me to write it.  As I started
to write, it rapidly appeared that it required to treat somewhat
thoroughly and broadly the aspect of cognitive synergy.  This actually
makes the major part of the document.  Please, take it as an initial
proposal though, and a proposal among many.  It is clear that for
cognitive synergy to be realized effectively, it requires a concerted
agreement across the creators (and users) of Hyperon, as well as an
ongoing exploration.

Because I want to make the point clear, let me quickly summarize the
main idea of cognitive synergy I develop in this document.  The main
idea is to replace handwritten heuristics appearing throughout the
AS-MOSES code base by explicit logically formulated queries.
Heuristics exist because the optimal choices are too difficult to
make.  That is precisely where Hyperon comes to the rescue.  When a
choice is too difficult and too impact-full, instead of offering a
hardwired, rigid and ultimately incomplete solution, poke a hole, look
outside and ask Hyperon, in its global wisdom, for help.

\section{AS-MOSES, a brief history}

MOSES~\cite{Looks06abstractcompetent}, which stands
for \emph{Meta-Optimizing Semantic Evolution Search}, is an
evolutionary program learner initially developed by Moshe Looks.  It
takes in input a problem description, and outputs a set of programs
supposed to solve that problem.  A typical example is the problem of
fitting data, in that case the problem description may be a table of
inputs and outputs alongside a fitness function measuring how well a
candidate fits that data.

I believe the primary motivation behind the creation of MOSES came
from the desire to adapt existing EDA (Estimation of Distribution
Algorithm) methods to evolve programs instead of bitstrings.  Upon
investigating that space, Moshe, with the help of Ben, discovered that
by combining a few tricks, evolving programs using EDA was actually
competitive.  We'll come back to these tricks but, let me say that, as
with anything, these tricks tend to only work well under some
assumptions, which will attempt to recall as well.

At some point Moshe left, and Linas Vepstas and I took over the MOSES
code base, added a number of improvements such as stochastic
hillclimbing, crossover, diversity pressure and more.  We, however
never got a chance to improve the EDA part (remaining stuck in an
somewhat incomplete stage), which is probably regrettable because, I
believe, if done well, this method has a lot to offer.  I do not
expand much about improving EDA in this document, only offers a few
hints.  Maybe this subject deserves a future document of its own.

In terms of performances, MOSES did achieve good results for its time
for evolving programs restricted to Boolean expressions.  It performed
poorly for programs involving floating point numbers (in spite of
Moshe inventing a clever representation of numbers amenable to linkage
learning).  Later on, it was also used to evolve programs controlling
agents to perform imitation learning in virtual environments, with
some amount of success.

Initially, the target programming language (i.e. the representational
language of the candidates being evolved) supported by MOSES was its
own thing, called Combo and described by Moshe as "Lisp with a bad
haircut".  Later on, as OpenCog Classic developed, the need to
integrate MOSES more deeply into OpenCog Classic came to be.  And the
task of replacing Combo by Atomese, the language of OpenCog Classic
(the equivalent of MeTTa for Hyperon) was initiated, carried out, for
the most part, by Kasim Ebrahim.  The result was called AS-MOSES, for
AtomSpace-MOSES.  That endeavor was never completed because Kasim
left, and the development effort was shifting from OpenCog Classic to
Hyperon anyway.

The repositories of MOSES can be found here~\cite{MOSES} and that of
AS-MOSES can be found there~\cite{ASMOSES}.  There are quite similar.
The main difference is that AS-MOSES contains some code for evolving
Atomese programs beside Combo.  The AS-MOSES code base is, in some
limited respects, a bit cleaner, but also larger due to the additional
Atomese support.  In the rest of the document I will often mention
MOSES while meaning either MOSES or AS-MOSES, or MOSES Classic to
denote versions coming from OpenCog Classic as opposed to future
Hyperon versions.

\section{The Goal of the Port}

I believe the goal of the port should not be to verbatimely reproduce
AS-MOSES inside Hyperon.  The goal, in my opinion, should be\\

\emph{To create a sufficiently open-ended program learning framework
that integrates well with the rest of Hyperon to enable some form of
cognitive synergy}.\\

Indeed, even though AS-MOSES contains important innovations that we
want to be ported, it essentially misses the cognitive synergy aspect.

Pragmatically speaking, cognitive synergy here means that if MOSES
gets stuck in the process of evolving programs, it can formulate a
request of help to the rest of Hyperon and take advantage of it to
unstick itself.  Likewise, if other components of Hyperon are stuck,
they can formulate requests of help to MOSES and take advantage of it.
There are certainly many ways such cognitive synergy could be
realized, I will describe one that I like, or that at least I
understand, but ultimately how to do that well, i.e. producing synergy
as opposed to interference, remains an open question.

There is also another form of synergy, perhaps just as important, the
synergy between MOSES and its users.  This could for instance take the
form of integrating MOSES to the MeTTa LSP server.  More will be said
on that further below.

\chapter{Porting MOSES Classic}

\section{MOSES Secret Sauce}

Let me briefly recall the main tricks that contributed to make MOSES
competitive.

\begin{enumerate}
\item Reduce candidates to normal form.  That trick consists in
  applying rewriting rules to transform candidates into some (ideally
  unique) canonical form while preserving semantics.  For example
  \mintinline{scheme}{(and y x)} would become
  \mintinline{scheme}{(and x y)}, thus if MOSES generates both
  \mintinline{scheme}{(and x y)} and
  \mintinline{scheme}{(and y x)},
  after reduction they would both point to the same candidate.  This
  has the following advantages:
\begin{enumerate}
\item Avoid re-evaluating syntactically different, yet semantically
  identical candidates, saving resources in the process.
\item Candidates that are more consistently formatted are also easier
  to recombine meaningfully.
\item Increase syntactic versus semantics correlation.  It is possible
  to design reduction rules so that candidates that are syntactically
  similar after reduction are also more likely to be semantically
  similar.  This has the effect of making the fitness landscape less
  chaotic, therefore less deceptive.  The Elegant Normal Form, used in
  the reduction engine for Boolean expressions, happens to have such
  property.  It also happens to have the property of not utterly
  destroying the hierarchical structure of an expression, making it
  more suitable for modular recombination.  This is aligned with of
  one of the core assumptions underlying MOSES success, which is that
  the problems we care about tend to be decomposable.
\end{enumerate}
\item Locally vectorize the search space.  Given an exemplar
  candidate, MOSES generates a program subspace around that candidate,
  called a deme.  Such subspace happens to be a vector space, thus
  amenable to a battery of optimization techniques, including EDAs,
  hillclimbing, cross-over and more.  Upon optimizing a deme, MOSES
  collects promising candidates that can subsequently be used as
  exemplars to spawn more demes.
\item Vectorize the fitness.  Instead of reducing the fitness to a
  single number, the fitness can be represented as a vector of
  components, thus providing some support for multi-objective
  optimization.  For example, MOSES can be asked to retain only the
  Pareto front of a population.  Additionally, diversity pressure can
  be applied during search by taking into account the distance between
  these vectorized behavioral-like scores.  Typically, in MOSES
  components would represent the fitness of the candidate for each
  data point, as opposed to just its aggregated fitness.
\end{enumerate}

As always, it was observed that these tricks could speed-up evolution
in some situations and slow it down in others.  Choosing the right
hyper-parameters for a given problem was often a difficult task.

\section{Porting MOSES's Tricks to Hyperon}

What is enumerated above is not the full set of tricks MOSES used, but
constitute a good starting point for porting to Hyperon.  Let me
provide some high-level guidance in that respect.

\begin{enumerate}
\item Reduce candidates to normal form.  I believe this can be
  elegantly ported using either the native MeTTa pattern-based
  interpreter, or the chaining technology developed here.  One
  technical difficulty when doing so is that program variables can be
  confused with program holes during reduction (via evaluation or
  chaining).  Using De Bruijn index would be one way to address that
  problem.  Them, it requires to decompose the Elegant Normal Form
  algorithm as an explicit set of rewriting rules.  I believe it is
  worth the effort because it may then offert more flexibility and
  extendability.  Indeed, some problems require weaker or stronger
  forms of reduction.  By breaking down a monolithic implementation
  into rules, one can more easily assemble subset of rules to control
  the reduction strength on a per situation basis.  Some of these
  rules can more easily be generalized to domains beyond Boolean
  expressions.  Finally, reframing reduction as a form of reasoning
  opens the door to more flexible hybridization between evolution and
  reasoning.
\item Locally vectorize the search space (aka representation building
  in MOSES).
\begin{enumerate}
\item Even though vectorizing the search space is a convenient and
  powerful way to represent a space to optimize, the port should also
  explore beyond that.  Vectorizing has also drawbacks, one being that
  what is learned in a local representation is not necessarily easy to
  transfer to other representations.  Indeed, the wisdom accumulated
  while searching a deme should ideally be transferable to other
  demes.  For instance, in MOSES Classic, EDA was taking the form of
  learning a Bayesian Network over the components of that vector
  space.  But since the semantics of each component was not the same
  for other vector spaces, the knowledge accumulated during the
  optimization of a deme would not be directly transferable to other
  demes.  I thus recommand to port that aspect but also to explore
  other, perhaps more global, representations (as the iCog team has
  already started).
\item Also, the Bayesian learning phase did not take into account the
  confidence of the probabilities learned, which in turn made it
  difficult to properly balance exploration and exploitation during
  the sampling phase.  PLN, which has a native support for confidence,
  could potentially be used as a replacement.  There are certainly
  also promising avenues to explore to combine EDA and hillclimbing.
  I believe the litterature already contains such research.
\item I should mention that when vectorizing a program subspace, there
  is an inherant tension between expressivity and regularity.  The
  more expressively dense a representation is, the more deceptive it
  likely is as well, so vectorizing should be flexible and easily
  reprogrammable.  Among the set of possibilities, perhaps the
  following paper could be relevant~\cite{Alon2019}.  Likewise, there
  is the problem of minimizing over-representation in that vector
  space, meaning that ideally one vector should correspond to one
  candidate.  MOSES contains heuristics to minimize that form of
  over-representation by cleverly using, albeit heuristically as well,
  the reduction engine to discard dimensions at representation
  building time.
\end{enumerate}
\item Vectorize the fitness.  I think MOSES did a good job there and
  it can probably be ported as it is.  There are a number of diversity
  distances that could be ported as well.  Although to be perfectly
  clear, as every hyper-parameters in MOSES, choosing when and when
  not to use such diversity pressures is difficult.  For instance,
  retaining only the Pareto front would speed up the search for some
  problems, but slow it down for some others.  Same thing for
  diversity pressure and other diversity related hyper-parameters.
  Regarding fitness functions, there is something that can be pushed
  to the next level though, to not only decompose fitness into
  multiple components but to make the whole fitness a clear box,
  amenable to analysis and reasoning, as opposed to an opaque black
  box only used for candidate evaluation.  See Section
  \ref{sec:blackbox-clearbox} for a discussion on the matter.
\end{enumerate}

\chapter{Cognitive Synergy}

\section{Realizing Cognitive Synergy between MOSES and Hyperon}
How can MOSES be helped by Hyperon?  First, any hyper-parameters
controlling MOSES, such as for instance the portion of evaluations
that should be allocated to search any particular deme, can
potentially be tuned by Hyperon.  Second, a big help would probably go
to the optimization phase.  This is after all where most of the
computational resources are spent.  Selecting the right optimization
algorithm for the right problem is an example (though may fall under
the hyperparameter tuning aspect mentioned above).  If the
optimization algorithm is EDA-based, an important avenue for help is
in modelling the fitness landscape, which, as mentioned above could
use PLN.  There is also the problem of transfering knowledge across
demes, and ultimately across problems as well.  For instance ideally
the wisdom accumulated to effectively sample a deme should not be
thrown away when a new deme is created.

So how to concretely achieve that?  The precise answer needs research
and development, but some directions can be provided.  The main idea
would be to formulate in logic, PLN or whichever logic is adequate,
the relationships between problems and various aspects of MOSES.  For
instance in the context of hyperparameter tuning, such formulation may
look like\\

\emph{If problem $\pi$ has property p, then hyperparameter f should be within
range r to achieve greater than average performance with probably
$\rho$}.\\

Then given such statements, a planner would be able to set the
hyperparameters before launching MOSES on a particular problem.  The
same idea would apply to finer aspects of MOSES, such as the
optimization phase.  In the case of EDA-based optimization, such
statement could look like\\

\emph{If deme d has property p, then if candidate c contains operator $o_1$ at
location $l_1$ and operator $o_2$ at location $l_2$, it is likely to
be fit with probably $\rho$}.\\

Then the EDA procedure could, at particular phases, query the
atomspace for such wisdom.  If none is retreived then it would proceed
as usual, but if some is then it would be able to take advantage of it
and diverge from its default behavior.

How these logical statements could be acquired is too broad of a
subject to be properly treated here.  But in essence this would also
be delegated to Hyperon, by providing traces of instances of MOSES
solving past problems, asking Hyperon to mine those traces to discover
patterns, and populate a space of logical statements reflecting these
patterns, which would in turn accelerate MOSES in the next runs.

An alternate, though somewhat equivalent in spirit way, suggested by
Ben Goertzel a while ago would be to formulate tasks as calls to a
universal sampler (a function able to sample any distribution layered
with any constraint), called SampleLink~\cite{SampleLink}.  The
difficulty then comes does down to providing an implementation of such
universal sampler that can utilize Hyperon accumulated wisdom.

Generally speaking it means is that the way MOSES (or SampleLink)
needs to be implemented should follow the rule\\

\emph{When the decision is hard to make, ask Hyperon for help}.\\

For instance if there is a conditional in the MOSES code

\begin{minted}[mathescape]{scheme}
(if C B_1 B_2)
\end{minted}

and \texttt{C} happens to be difficult to establish, then \texttt{C}
should be formulated as a query to Hyperon.  In other words, anything
that is too hard for MOSES alone should be delegate to Hyperon.  When
a trace of a run of MOSES is recorded, it should also record these
queries and their results, because this can inform Hyperon about what
needs to be improved.  If the query corresponding to
condition \texttt{C} often came back unanswered, or answered with low
confidence, it gives a cue to Hyperon that, in order to better help
MOSES, it must find ways to better answer that query in the future.

I have left undefined the notion of \emph{query to Hyperon}, but for
starter one can simply have in mind a \emph{pattern matching query},
because Hyperon should hopefully be hyper optimized to fulfill these.
Thus the idea is\\

\emph{If Hyperon already knows how to help, then it can help on the spot at
almost no cost by answering the query.  Otherwise, it does not help,
thus MOSES defers such a default behavior, but Hyperon can keep a
trace of the interaction for future improvements.}\\

Such notion of \emph{query to Hyperon} can be extended by for instance
using the chainer.  In that case \emph{pattern matching query} would
be replaced by \emph{reasoning}, or perhaps \emph{shallow reasoning}.
It means however that the effort spent in such reasoning needs to be
properly controlled.  That could be done by for instance by
guarantying some temporal upper bound as to make reasoning about the
overall efficiency of MOSES easier.\\

Next, how can MOSES helps Hyperon? Any problem that can be formulated
as being solved by finding programs fulfilling some fitness can likely
be solved by MOSES.  That may or may not encompass any problem.  Of
course MOSES will tend to perform better on some problems and worse on
others.  So, determining whether MOSES can help is some particular
situation amounts to being able to formulate the proper fitness
function and then evaluate how efficient MOSES can be on that fitness
function.  Thus Hyperon should progressively accumulate knowledge to
be able to estimate how well MOSES can come up with a solution for a
particular problem, given some amount of available resources.

One may note that MOSES should be able to discover pattern inside its
own traces, in the manner described in [How can Hyperon helps
MOSES?](how-can-hyperon-helps-moses?).  Thus by applying MOSES at the
meta-level, MOSES could in fact help itself.

\subsection{Cognitive Synergy between MOSES and Humans}

Integrating MOSES in the MeTTa LSP server could be one way to enable
some form of synergy between MOSES and MeTTa programmers.  Imagine for
instance a programmer is writing some function in MeTTa, has some
examples of how it should behave but no algorithm in mind yet.  The
programmer would be able to invoke MOSES via the MeTTa LSP server and
get a list of candates solutions in return.

The other direction, human programmer helping MOSES, should be
possible as well.  Upon launching MOSES, the user should be able to
interrupt it, query its state, inspect its memory content, and even
modify it as to change the direction of the search.

\subsection{The Case for a Common Language}

As I said there are multiple ways cognitive synergy can be realized.
What I am going to present is simply the use of a common language to
communicate between parts of Hyperon.  Let me reuse the old notion of
*mind agent* from OpenCog Classic, as cognitive process that would
operate within Hyperon.  Mind agents would be for instance MOSES, the
backward chainer, ECAN, etc.  So the idea is that all mind agents
share a common language to formulate requests of help to each others.
Which brings the question of what language to use.

Of course the answer is MeTTa, but there are many ways MeTTa can be
used.  So more specifically, for starter, I suggest to borrow standard
constructs from Dependently Typed Languages, such as dependent sums
and products.  To be clear, I am not necessarily advocating that we
use such a language in the long term, even though I believe it is a
good start due to its expressive power and popularity.  But it also
has drawbacks, probably the main one being that it is based on a crisp
typing relationship.  So we may for instance want to replace that by
some probabilistic extension as explored by Jonathan Warrell, Greg
Meredith and Mike Stay.  Of course, one ought to mention PLN as a
potential candidate as well.  PLN was in fact the primary candidate
for such common language back in the OpenCog Classic days, but with
the recent developments of probabilistic dependent types that question
requires reconsiderations.  PLN has also some drawbacks, one being
that, at least as formulated in the PLN book, it is non-constructive,
unlike dependent types.  But it is conceivable that a future version
of PLN, built on top of such probabilistically dependently typed
languages, may become once again that common language.  These
questions will need to be carefully re-examined as conceptual and
technical progress are being made.  For now let me simply explain how
such common language, as a regular dependently typed language crafted
for MeTTa, can be used for cognitive synergy.

The idea would be that when calling a mind-agent, the description of
the problem to solve is provided in that common language.  So, it does
not matter if the mind agent is MOSES, the backward chainer, or
something else, the query would essentially look the same.  That
approach is reminiscent to Ben Goertzel `SampleLink` idea, but
materialized somewhat more conventially, using type theoretic query
answering at the center rather than sampling.  Which does not exclude,
far from it, to re-introduce explicit forms of sampling later on.  So
the basic format for calling such mind agent would be as follows

\begin{minted}[mathescape]{scheme}
(MIND_AGENT HYPER_PARAMETER QUERY)
\end{minted}
where
\begin{itemize}
\item \mintinline{scheme}{MIND_AGENT} is a MeTTa function, such as \mintinline{scheme}{moses},
for MOSES, or \mintinline{scheme}{bc} for the backward chainer, etc.
\item \mintinline{scheme}{HYPER_PARAMETER} is a data structure containing all the
  hyper-parameter for the call.  That structure would contain for
  instance the effort to allocate, various default heuristics, how
  much complexity pressure to apply, pointers to spaces containing
  meta-knowledge, etc.
\item \mintinline{scheme}{QUERY} is the query itself, a description of the problem to
  solve.  That description does not necessarily have to be big and
  complex because it can take advantage of a vocabulary defined in
  spaces referenced inside the \mintinline{scheme}{HYPER_PARAMETER} structure.
\end{itemize}

For instance, if one wishes to evolve a program computing a binary
function that fits a certain data set, one may express that with

\begin{minted}{scheme}
(moses MOSES_HYPER_PARAMETER
       (: $cnd_prf (Σ (-> Bool Bool Bool) (FitMyData $fitness))))
\end{minted}
where
\begin{itemize}
\item \mintinline{scheme}{(-> Bool Bool Bool)} is the type signature of the candidate we are
  look for.
\item \mintinline{scheme}{FitMyData} is a parameterized type representing a particular custom
  fitness measure.  For the query to be understood, \mintinline{scheme}{FitMyData} must
  be defined in a space referenced inside \mintinline{scheme}{MOSES_HYPER_PARAMETER}.
\item \mintinline{scheme}{$fitness} is a MeTTa variable representing a hole in the query to
  be determined by MOSES for each candidate, corresponding to the
  actual fitness score of that candidate.
\item \mintinline{scheme}{Σ} is the Sigma type, aka dependent sum, expressing the existence of
  such candidates in a constructive manner.
\item \mintinline{scheme}{$cnd_prf} is a hole representing an inhabitant of that sigma type,
  which, upon answering the query, should contain both the candidate
  and the proof that this candidate fulfills the query.  If more than
  one such candidate exists, then the result should be a superposition
  of the inhabitants.
\end{itemize}

In that example \mintinline{scheme}{FitMyData} is used to hide the complexity of
the query, it does mean though that a space containing the definition
of
\mintinline{scheme}{FitMyData} must be provided in the hyper-parameter, and other
mind-agents will need to have access to that space to fully unpack the
meaning of \mintinline{scheme}{FitMyData}.  More self contained definitions can also be
provided by a structured type instead of a mere symbol referring to a
predefined type.  How exactly that structured type would look like is
beyond the scope of that document and does not matter too much.  All
that matters is that the resulting type follows the type signature
required by \mintinline{scheme}{Σ}, which in that specific example would be

\begin{minted}{scheme}
(-> (-> Bool Bool Bool) Type)
\end{minted}
meaning that \mintinline{scheme}{FitMyData}, or whatever equivalent structured type, must
describe a type constructor that takes a binary boolean function and
returns a type.  This is a common way to represent predicates in
Dependently Typed Languages.

Maybe one realizes that MOSES is in fact inadequate to discover such
candidate, thus may attempt to call the backward chainer instead

\begin{minted}{scheme}
(bc BC_HYPER_PARAMETER
    (: $cnd_prf (Σ (-> Bool Bool Bool) (FitMyData $fitness))))
\end{minted}
As you can see the call is almost the same, only the function being
called and its hyper-parameters are different, the query is identical.

But the use of that common language does not stop here as mind agents
can use such querying format internally.  Let's say for instance that
MOSES, within the course of its execution has an important decision to
make, which could be for instance: should I search a given deme more
deeply, or abandon that deme and create a new one?  A sketch of the
code could like like:

\begin{minted}{scheme}
(= (moses $hps $query)
   BODY ...
         (if (continue-search-deme $deme)
             SEARCH_DEME
             CREATE_NEW_DEME)
         ...)
\end{minted}
The idea is that instead of having \mintinline{scheme}{continue-search-deme} make
that decision in isolation, MOSES can formulate that question to
Hyperon.  If Hyperon knows the answer, then MOSES can take advantance
of that knowledge, ortherwise it can defer to a default behavior.
What it means that is the code of \mintinline{scheme}{continue-search-deme},
instead of solely consisting of hardcoded heuristics, may contain
queries such as

\begin{minted}{scheme}
(bc BC_HYPER_PARAMETERS QUERY_ABOUT_DEME_CONTINUATION)
\end{minted}
Here the backward chainer is used as example because it is assumed to
be somewhat unversal, maybe we want to have an even more universal
access point such as \mintinline{scheme}{hyperon}, this looking even closer to the
SampleLink idea.  Regardless, what will typically happen is that
such query will be parameterized to have a minimal cost, to not slow
down MOSES in its process flow.  For instance the depth of reasoning
used to answer that query could be null or almost null, keeping the
reasoning shallow and inexpensive, if any.  That way, if Hyperon knows
the answer it can help MOSES right away, otherwise, a record of that
innability to answer can be kept and used as feedback to incentivize
Hyperon to learn to succeed in the future.

One could envision a scenario where MOSES is being called on a series
of problems, while in the background other mind-agents operate some
forms of meta-learning and meta-reasoning to build-up the knowledge to
eventually help MOSES.  This can be illustrated as follows:

{\small
\begin{minted}{scheme}
(moses HP1 QUERY1)  |  Meta-learning ...         ; <- Fails to help
(moses HP2 QUERY2)  |  Meta-learning ...         ; <- Fails to help
(moses HP3 QUERY3)  |  Meta-learning ...         ; <- Fails to help
(moses HP4 QUERY4)  |  Meta-learning (discovery) ; <- Fails to help
(moses HP5 QUERY5)  |  Meta-learning ...         ; <- Hyperon succeeds!
\end{minted}
} During the first four runs, Hyperon fails to help MOSES at that
particular decision point (such as deme continuation).  During the
fourth run the knowledge to help is finally discovered, leading
Hyperon to guide MOSES during the fifth run by making it take the
branch with the likely better outcome.

\section{Black Box vs Clear Box}
\label{sec:blackbox-clearbox}

The most basic way to use a fitness function is as a black box that
can score candidate solutions and nothing more.  That is likely the
only way when it is written in a foreign programming language such as
C++.  However, as soon as it is written in MeTTa it becomes a clear
box.  Such fitness can now be analyzed and reasoned upon.  The
possibilities this offers are limitless.  Just to give a simple
example, one could for instance invoke a reasoner to come up with a
fitness estimator that is less accurate but more efficient than the
original fitness, while guarantying properties such that for instance
the estimator pointwise dominate the original fitness, so that it will
never under-evaluate good candidates, etc.

I mentioned earlier that MOSES was able to handle multi-objective
fitness, it was in fact the only available transparency with respect
to the fitness function MOSES had.  Using MeTTa to describe such
fitness would allow to go much beyond that.

\section{Program Evolution as a Form of Reasoning}

One way to realize cognitive synergy is to never really leave the
logical side.  That is, having MOSES explicitely operate as a form of
reasoning.  I am not necessarily advocating for that, because it has a
number of drawbacks, but it is likely the way I would do it if I were
tasked to do the MOSES port myself.  The main drawbacks are
\begin{enumerate}
\item the unfront cost of formulating evoluationary learning as a form of
reasoning;
\item the run-time cost of doing evolution using reasoning.
\end{enumerate}
The main benefit in my view, is that it enables, at least in
potential, the deepest levels of cognitive synergy one can hope to
achieve.  Besides, over time the run-time cost can be mitigated by
\emph{schematizing} (as Ben Goertzel likes to say), or \emph{specializing} (as
Alexey Potapov likes to say) the parts of MOSES that require the least
amount of synergy with the rest of Hyperon.

Perhaps a hybrid approach can be considered from the get go, where
MOSES is partly implemented in a functional way, and partly
implemented as an explicit form of reasoning.  The non-determinism of
MeTTa can actually make these distinctions somewhat blurried.

The general idea of framing evoluation as a form of reasoning is that
the problem of finding good program candidates is directly formulated
as a query in logic, then MOSES merely provides an efficient inference
control mechanism to fulfill such query, and by that discover such
program candidates.

There are at least two ways to formulate such query, in a
non-constructive vs constructive way.  Explaining in depth the
difference between the two is probably outside of the scope of that
document but for our concern it suffices to say that in constructive
mathematics proving an existential statement, such as $\exists x
P(x)$, garanties that there is a way to construct an object $a$ such
that $P(a)$ is true.  While in non-constructive mathematics one could
indeed prove the existence of such an object without ever having to be
able to construct it.  Due to that major difference, such query will
take a different form if it is done in a constructive versus a
non-constructive way.

\subsection{Constructive Way}

In the constructive case a statement leading to finding a program
candidate may merely look like

$$\exists x\ \texttt{GoodFit}(x)$$
Then finding a proof of such statement will provide a program
candidate.  To find more candidates one may simply finding more
proofs.

\subsection{Non-constructive Way}

In the non-constructive way, it is also possible to use reasoning to
find program candidates, but the candidate must instead be represented
as a free variable inside the statement, and the reasoning must be
able to instantiate that variable during reasoning.  At the end of the
reasoning process, the candidate may be absent from the proof, but
present inside a refined version of the statement.  For instance such
statement may initially look like

$$\texttt{GoodFit}(x)$$
Then finding a proof will result in a program candidate appearing in
the now grounded statement

$$\texttt{GoodFit(CANDIDATE)}$$
The backward and forward chainers of Hyperon support both,
constructive and non-constructive, ways.  Thus it is not immediately
obvious which way is best.  Ultimately we will have to experiment with
both ways to really know.

\subsection{Type Theoretical Way}
Perhaps contrary to popular beliefs, the type theoretical way can be
used both constructively and non-constructively, although it
particularily shines when used constructively.

We will give below an example of how to use dependent types, and more
specifically the Sigma type, Σ, to represent such existentially
quantified statements and to discover program candidates.

\subsubsection{GoodFit as Type Predicate}

The standard way to represent existential quantification with
dependent types is via the Sigma type, Σ, which can be defined in
MeTTa as follows

\begin{minted}{scheme}
(: MkΣ (-> (: $p (-> $a Type))
           (: $x $a)
           (: $prf ($p $x))
           (Σ $a $p)))
\end{minted}
where
\begin{itemize}
\item \mintinline{scheme}{$p} is a predicate type,
\item \mintinline{scheme}{$a} is the domain of \mintinline{scheme}{$p},
\item \mintinline{scheme}{($p $x)} is \mintinline{scheme}{$p} applied to an element \mintinline{scheme}{$x} of \mintinline{scheme}{$a},
\item \mintinline{scheme}{$prf} is a proof of \mintinline{scheme}{($p $x)},
\item \mintinline{scheme}{(Σ $a $p)} is the Sigma type expressing that there exists an
  element of \mintinline{scheme}{$a} which satisfied property \mintinline{scheme}{$p}.
\end{itemize}
In our case the property will be the goodness of fit.  However,
because we want to be able to quantify that fitness, the predicate is
parameterized by the fitness score.  In the end the type we are
looking for is something like

\begin{minted}{scheme}
(Σ (-> Bool Bool Bool) (Fit 0.7))
\end{minted}
where \mintinline{scheme}{(-> Bool Bool Bool)} represents the domain of the candidates,
in this case, binary Boolean functions, and \mintinline{scheme}{(Fit 0.7)} represents the
predicate expressing the class of candidates that are fit with degree
of fitness 0.7.  Thus \mintinline{scheme}{GoodFit} has been replaced by the parameterized
type \mintinline{scheme}{(Fit FITNESS)}.

A proof for such type may look like

\begin{minted}{scheme}
(MkΣ (Fit 0.7) (\ $x $y (and $x $y)) PROOF_OF_FITNESS)
\end{minted}
where \mintinline{scheme}{\} represents a lambda abstraction.  Note
that, as of the time of writing this document, \mintinline{scheme}{\}
is not part of the MeTTa syntax.  However, JeTTa a Java based MeTTa
transpiler, seems to already supports it.  Regardless, in my made-up
syntax
\begin{minted}{scheme}
(\ $x $y (and $x $y))
\end{minted}
would be typed \mintinline{scheme}{(-> Bool Bool Bool)},
while
\begin{minted}{scheme}
(\ $x (\ $y (and $x $y)))
\end{minted}
the curried version, would be typed \mintinline{scheme}{(-> Bool (-> Bool Bool))}.

One can verify that each argument is an inhabitant of the argument
types of \mintinline{scheme}{MkΣ}, meaning
\begin{itemize}
\item \mintinline{scheme}{(: (Fit 0.7) (-> (-> Bool Bool Bool) Type))}
\item \mintinline{scheme}{(: (\ $x $y (and $x $y)) (-> Bool Bool Bool))}
\item \mintinline{scheme}{(: PROOF_OF_FITNESS ((Fit 0.7) (\ $x $y (and $x $y))))}
\end{itemize}
\mintinline{scheme}{PROOF_OF_FITNESS} is left unspecified for efficiency reasons.  We
will see in fact that we can inject computations in that reasoning
process, thus not necessarily framed as a form of reasoning, to speed
up some aspects of it.  Note that a proof of
\mintinline{scheme}{(-> Bool Bool Bool)}
is a program, and the reasoning process will be able to build that
program in the same manner that it can build a proof of fitness for
that program.

A prototype of such evolutionary program learner framed as a reasoning
process can be found here~\cite{EvoReason}.  Beware though that a few
of things have been changed.
\begin{enumerate}
\item In order to quote the programs, to avoid spontaneous reduction by
the MeTTa interpreter over a reducable MeTTa expression such as

\begin{minted}{scheme}
(and True False)
\end{minted}
the vocabulary does not include MeTTa built-ins such as
\mintinline{scheme}{and}, \mintinline{scheme}{True}, etc.  Instead new symbols are introduced using unicode,
so that \mintinline{scheme}{and} in MeTTa becomes \textbf{and} (in
bold font using unicode) in the reasoning process, etc.  This could
perhaps be avoided by a clever use of the \mintinline{scheme}{quote}
built-in, or by implementing the chainer in Minimal MeTTa.  But for
the sake of simplicity this prototype uses this unicode trick instead.
\item Similarily, to avoid spontaneous unification from taking placed
during reasoning, variables inside programs are replaced by De Bruijn
indices.  So for instance
\begin{minted}{scheme}
(\ $x $y (and $x $y)))
\end{minted}
would become
\begin{minted}{scheme}
(\ z (s z) (and z (s z))))
\end{minted}
where \mintinline{scheme}{z} is the first De Bruijn
index, \mintinline{scheme}{(s z)} is the second, etc\footnote{Note
that traditionally, De Bruijn indices do not need to be mentioned
after a lambda, so for instance $\lambda x.x$ merely becomes $\lambda
0$.  But in MeTTa, where currying is not automatic, there is a need to
represent a lambda over multiple variables, which the suggested syntax
solves.  I must say that in practice I have found that keeping De
Bruijn indices after the lambda tends to enhance readability anyway.}.
\item To avoid exacerbating combinatorial explosion, lambda abstraction is
actually completely eliminated.  So instead of introducing program
variables (i.e. De Bruijn indices) via lambda abstraction, these are
added to the environment at the beginning of reasoning.  As a result
instead of evolving a program with the type
signature
\begin{minted}{scheme}
(-> Bool Bool Bool)
\end{minted}
the following assumptions \mintinline{scheme}{(: z Bool)}
and \mintinline{scheme}{(: (s z) Bool)} representing the types of the
first two arguments of the program to evolve, are added to the
environment.  And thus the type signature of the program to evolve is
replaced by \mintinline{scheme}{Bool}.  Note that this gymnastic is
actually exactly what the chainer does when encountering a lambda
abstraction, so it is a rather natural trick.  The difference is that
it is done before reasoning and never during reasoning, because, as I
said, introducing lambda abstraction on the fly during reasoning
exacerbates combinatorial explosion.  Indeed, every new lambda
abstraction increases the possibilities of function applications, and
every new function application provides more body for lambda
abstraction.  This situation becomes rapidely unmanageable, thus why
we try to avoid it.  It may be that at some point reasoning becomes so
efficient that lambda abstraction can be re-introduced.
\end{enumerate}

\subsection{Reasoning Efficiently}

Of course framing learning as a form of reasoning, as elegant as it
may be, is only worth it if it results in an efficient process.  There
are at least two ways this can be accomplished:
\begin{enumerate}
\item Injecting regular computation in the reasoning process.  The idea
is to outsource some proof obligations to some external computational
processes that we trust.  For instance, let's say we want to prove
that 2 + 3 = 5.  One way to do this is to use reasoning exclusively,
progressively transforming
\begin{itemize}
\item 2 + 3
\item to 1 + 4,
\item then to 0 + 5,
\item then finally to 5,
\end{itemize}
all that by manipulating the laws of equality and addition.  Even
though this way is perfectly correct it is also costly.  Another way
to do this is to query the CPU with an ADD instruction with 2 and 3 as
arguments, get the results, and trust that the resulting equation is
indeed true.  This is how \mintinline{scheme}{PROOF_OF_FITNESS} would
be typically obtained in our evolutionary programming case.  In fact
in the prototype referenced earlier such proof is
labelled \mintinline{scheme}{CPU} to convey the idea that ``it is true
because the CPU said so''.  Of course, in reality, more that the CPU
has to say so, the function that is part of that external
computational process has to be properly implemented, but we call
it \mintinline{scheme}{CPU} to capture that idea.
\item Leveraging inference control.  This is where the most speed-up can
   be obtained, but this is also the most difficult way.  It is one of
   those ``AGI-complete'' problems, one that if you achieve it, you
   likely achieve AGI.  A potential solution for this problem is
   almost the same as the one presented for cognitive synergy.  A
   record of past inferences is stored and analyzed to extract
   predictive patterns used to guide an inference control mechanism
   for subsequent inferences.  This would be the most generic
   solution.  But one can also consider more specific handcrafted
   solutions, by implementing good old heuristics.  Indeed, one can
   see a particular Genetic Programming algorithm as implementing such
   heuristics.  There are then two ways to do that
   \begin{enumerate}
   \item Substitute the chaining by a specific searching algorithm.  In
      that case the only remnant of a notion of reasoning is in the
      way the problem and the solutions are described (using a logical
      language as explained above).  The advantage is that one, by
      re-using existing search algorithms, can achieve speed quickly.
      The inconvenient is that it is somewhat inflexible to
      improvements, unless one is willing to learn search algorithms
      (which MOSES could in fact potentially do).
   \item Reframe the heuristics as a set of procedural rules guiding an
      explicit inference control mechanism.  For an example of how to
      do that, see the following experiment~\cite{InfControl}
      (beware that it is a very early stage prototype).  The
      inconvenient of that approach is that one needs to rethinking
      how to map the heuritics to that rule-based format.  The
      advantage is that it can then easily be combined with more
      generic meta-learning forms.  What that means is that the
      developer may initially provide a search heuristic, and then let
      Hyperon improve that heuristic on its own.
   \end{enumerate}
\end{enumerate}

\section{Modularity}

MOSES itself can be viewed as a collection of mind agents working in
concert.  So maybe the same format suggested in Section Cognitive
Synergy could be used as well.

Let us provide an example for the reduction engine.  A call to the
reduction engine in that format may look like

\begin{minted}{scheme}
(reduce RE_HYPER_PARAMETERS
        (: $cnd_prf (Σ (-> Bool Bool Bool)
                       (ENF (\ z (s z) (and (s z) z))))))
\end{minted}
where
\begin{itemize}
\item \mintinline{scheme}{reduce} is the entry point of the reduction engine.
\item \mintinline{scheme}{RE_HYPER_PARAMETERS} is the set of hyper parameters for \mintinline{scheme}{reduce}.
\item \mintinline{scheme}{(-> Bool Bool Bool)} is the type signature of the program to
reduce in normal form.
\item \mintinline{scheme}{ENF} is a binary predicate represented as
parameterized type, in that case with the following type signature
\begin{minted}{scheme}
(-> (-> Bool Bool Bool) (-> (-> Bool Bool Bool) Type))
\end{minted}
The first argument is a boolean binary function, the candidate not yet
normalized, and the second candidate is the same candidate in elegant
normal form.  Not that the predicate is curried in order to play
nicely with \mintinline{scheme}{Σ}.
\item \mintinline{scheme}{$cnd_prf} is an inhabitant of the sigma type, that is a dependent
pair with the candidate in elegant normal form as first argument,
which is merely \mintinline{scheme}{(\ z (s z) (and (s z) z))} in that
example, alongside the proof that it is indeed its elegant normal
form.  Just like the prototype of evolutionary programming presented
in Evolution Programming Section, the proof does not have to be fully
fledged and can be substituted by a obfuscated process that we simple
trust.  Likewise the `ENF` predicate can either provided as a
built-in, with perhaps some rules and axioms about it in an associated
knowledge base, or provided as a structured type containing in that
structure the meaning of what it is to be in elegant normal form.
\end{itemize}

I should mention that the same format could also be used to represent
data internally, which actually brings another avenue for fostering
cognitive synergy.  Indeed, having internal representations follow a
common language allows other mind agents to interact with that data as
well.  For example MOSES may represent its populations of candidates
as a collection of statements corresponding to the type relationships
following the same query format

{\small
\begin{minted}{scheme}
(: (\ z (s z) (not z)) (Σ (-> Bool Bool Bool) (FitMyData 0.2)))
(: (\ z (s z) (or z (s z))) (Σ (-> Bool Bool Bool) (FitMyData 0.5)))
(: (\ z (s z) (and z (s z))) (Σ (-> Bool Bool Bool) (FitMyData 0.8)))
...
\end{minted}
} If that format may look overly verbose for the purpose of internal
storage, note that it might end-up being automatically compressed by
the MeTTa backend (I believe MORK would offer that for instance).  And
if not, one could always realize such compression explicitely by
contextualizing the representation, as demonstrated in a recent
experiment on Modal Logic~\cite{ModalLogic}.

\section{Beyond MOSES}

Since MOSES was invented, the field of evolutionary programming has
advanced and there is no reason not to incorporate these advances into
MOSES, as the iCog team in charge of porting MOSES to MeTTa have
already started to explore.  Likewise, the existing components of
MOSES can be improved.  As I already mentioned properly balancing
exploration and exploitation in the EDA implementations is going to be
important.  Other improvements should take place as well to help MOSES
move beyond Boolean expressions.

\section{Cognitive Synergy between MOSES and humans}

\subsection{Programming Assistance}

The advances of LLMs have brought advances in programming assistance.
These however have deficiencies tied to the lack of use of formal
methods to produce correct programs.  Before LLMs, programming
assistance technologies existed but were reserved to the obscure world
of Automated and Interactive Theorem Provers (ATP and IPT).  I believe
MeTTa and Hyperon are bringing forth an opportunity to marry these two
approaches into a coherent and efficient whole.  It is not clear to me
how this is to be done properly but there are efforts within the
SingularityNET Foundation that would help to turn that vision into a
reality.  I would cite for instance the work done by the Semantic
Parsing group, MeTTa-Motto as well as NARSE-GPT.

I am personally not following the world of LLM based programming
assistance, so I will not comment on that.  On the ATP/ITP side, the
use of the Language Server Protocol (LSP) seems to have become a
standard.  The MeTTaLog team happens to be developing an LSP server
for MeTTa.  So one could envision to integrate MOSES, as well as some
of its components such as the Reduction Engine, to that LSP server.

I image the following scenario.  A MeTTa programmer wants to implement
a function but has fragmented knowledge about it, maybe has partial
knowledge about its inputs and outputs, as well as partial knowledge
about its properties.  The programmer could enter a prompt containing
examples of that functions, as well as expected properties, either in
natural or formal language.  If provided in natural language the
programmer assistant would come up with a formal specification (using
for instance the type theoretic representation presented in this
document as target language).  Then, after letting the programmer
double check, would produce a query for MOSES, or whatever tool is
appropriate for the job, send that query to the LSP server, and wait
for it to return solutions to that query.  If that process is too
resource demanding to run locally, the ASI Foundation could offer
remote services.

\subsection{Other Forms of Cognitive Synergies}

As MOSES is running to attempt to solve problems, it would be good if
users can interact and guide MOSES.  For instance, one should be able
to pause MOSES, read the content of its memory, and possibly modify
that content.  Likewise, it would be convenient if users could place
breakpoints at various locations in its code, so that when MOSES
reaches these breakpoints, the user is given a change to overwrite the
decision that MOSES would take next.  NEXT: it seems this should be
generalized to MeTTa and Hyperon.

\chapter{Conclusion}

\bibliographystyle{splncs04} \bibliography{local}

\end{document}
